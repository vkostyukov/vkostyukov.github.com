%% Copyright 2013 BarD Software s.r.o
%
% This work may be distributed and/or modified under the
% conditions of the LaTeX Project Public License version 1.3c,
% available at http://www.latex-project.org/lppl/.

\documentclass[11pt,a4paper]{moderncv-xetex}
\moderncvtheme[grey]{papeeria}

\usepackage{xcolor}
\usepackage[a4paper,margin=1in]{geometry}

\firstname{Vladimir}\familyname{Kostyukov}
\title{Compiler Engineer}
\mobile{+7(923)1319766}
%\phone{}
\email{vladimir.v.kostyukov@gmail.com}
\extrainfo{\href{http://vkostyukov.ru}{http://vkostyukov.ru}}
\quote{\phantom{qwetyaskjdlljaskldjsdqweoisdakdkjhkajshdk}}

\nopagenumbers{}

\hypersetup{colorlinks,urlcolor=blue}

\begin{document}
\maketitle

\section{Summary}
\cvline{}{\emph{Software Engineer} with 3 years of experience developing compilers and VMs.
Creative and passionate programmer with in-depth knowledges of managed runtimes and compilers.
Excellent Java and C/C++ programming skills allied with experience in performance analysis and benchmarking.
}

\section{Skills}
\cvline{Programming Languages}{Fluent in \emph{Java} and \emph{Scala}\newline
Efficient in \emph{C/C++}\newline
Knowledges in \emph{JavaScript}, \emph{Python}, \emph{Perl}
}
\cvline{Scopes}{Programming Languages Design and Implementation, Performance Analysis, Object Oriented Programming, Functional Programming, Purely Functional Data Structures, Design and Analysis of Algorithms}

\section{Experience}
\cventry{2011/04 -- ...}{\textbf{Software Engineer}}{\href{http://intel.com}{Intel Corporation}}{Novosibirsk}{Russia}
    {I am a member of the \textbf{Managed Runtimes} team. My first project there was development (from scratch) a proof of concept
    PKCS\#11 Java Crypto Provider (5k LOC), which based on Intel IPP libraries. The developed prototype showed 6X speedup relative
    to the default Java implementation.\newline
    I am currently involved into development of the x86 Trace-JIT compiler (targeted to Intel\textsuperscript{\textregistered} Atom\textsuperscript{\texttrademark} Architecture) for Dalvik VM.
    More precisely, I am responsible for development of both back-end (including code generation and instruction scheduling) and middle-end (including data-flow analysis and optimizations on CFG) components of the compiler.
    \newline
    \textbf{Tools:} \emph{Linux Shell, Git, Gerrit, Bugzilla, GCC, Intel VTune, Intel TBB, Intel IPP}\newline
    \textbf{Keywords:} \emph{Performance Analysis, Compilers and Interpreters, JIT Compilation, Low-Level and High-Level Optimizations, Data-Flow Analysis, CFG Construction and Analysis, Instruction Scheduling, Registerization, Benchmarking, Debugging}\newline
    }
\cventry{2010/10 -- 2011/04}{\textbf{Software Intern}}{\href{http://intel.com}{Intel Corporation}}{Novosibirsk}{Russia}
    {As a member of \textbf{Compilers and Languages} group I was responsible for performance tracking and analysis of Intel Compiler
    for MIC platform (a GPUGP chip with up to 128 cores). I gained an Intel SSG Award for being a pioneer of Intel MIC compiler performance
    tracking (developed a Perl-based harness and ported initial four workloads from NVidia CUDA SDK).\newline
    \textbf{Tools:} \emph{Intel Compiler Collections, Perl, Intel VTune, Linux Shell}\newline
    \textbf{Keywords:} \emph{Performance Analysis, GPU Offload, Benchmarking, Multithreading and Concurrency, Synchronization, Parallel Algorithms, Lock-Free Algorithms}\newline
    }
\cventry{2010/07 -- 2010/08}{\textbf{Summer School Intern}}{\href{http://intel.com}{Intel Corporation}}{Novosibirsk}{Russia}
    {I was working in \textbf{Java Xeon} team on the analysis of bottlenecks in the SPECjvm2008.serial test at Intel’s modern architectures. 
    The suggested solution (based on reducing number of stack frames) showed up to 50\% speedup on WSM-EX platform (in a multithreaded mode).\newline
    \textbf{Tools:} \emph{Linux Shell, JDK, GCC, Intel VTune, Vim, SPECjvm2008, Eclipse}\newline    
    \textbf{Keywords:} \emph{Java Performance Analysis, Benchmarking, Concurrency, JMM, JVM Internals, Serialization}\newline
    }
\cventry{2007/10 -- 2010/07}{\textbf{Technician}}{\href{http://en.altstu.ru}{Altai State Technical University}}{Barnaul}{Russia}
    {While working in IT department, I was responsible for maintaining network environment of university campus. I also was leading a 
    technical support team of ACM ICPC NEERC.\newline
    \textbf{Tools:} \emph{Linux Shell, Clonezilla}\newline
    \textbf{Keywords:} \emph{Scripting, OS Cloning, Network Administration}
    }

\section{Projects}
\cventry{la4j}{\textbf{Linear Algebra for Java}}{\href{http://la4j.org}{http://la4j.org}}{}{}
    {The la4j is a lightweight and 100\% Java library that provides Linear Algebra primitives and algorithms. It is highly popular 
    sparse/dense matrix library, which combines both fluent API and good performance.\newline
    \textbf{Tools:} \emph{Java SE, Eclipse, Maven, jUnit, Git, Travis-CI}\newline
    \textbf{Keywords:} \emph{Linear Algebra, Math, API Design, TDD, Design Patterns, Open Source}\newline
    }
\cventry{Quipu}{\textbf{Quipu Programming Language}}{\href{http://esolangs.org/wiki/Quipu}{http://esolangs.org/wiki/Quipu}}{}{}
    {The Quipu is an Esoteric programming language inspired by <<talking knots>> – recording devices historically used by Incas.
    It is a <<believed Turing-complete>> language, which means author believes that the language is Turing-complete, but no formal proof was provided.\newline
    \textbf{Tools:} \emph{Scala}\newline
    \textbf{Keywords:} \emph{Source Code Parsing, Interpretation}
    }

\section{Education}
\cventry{2006\,--\,2011}{\textbf{Master of Science} in CS}{\href{http://en.altstu.ru}{Altai State Technical University}}{Barnaul}{}
  {Master's thesis: \emph{"Distributed monitoring and dispatching system of the processes in heterogenous environment"}.\newline
  Grade: 95/100}

\section{Certificates}
\cventry{2010}{\textbf{IPPP-2-12}}{Intel Parallel Programming Professional}{}{}{}
\cventry{2010}{\textbf{HPC School 2010}}{Participant Certificate}{}{}{}
\cventry{2010}{\textbf{Intel Summer School 2010}}{Participant Certificate}{}{}{}
\cventry{2010}{\textbf{Intel Winter School}}{Participant Certificate}{}{}{}

\end{document}
